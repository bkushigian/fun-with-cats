%%%%%%%%%%%%%%%%%%%%%%%%%%%%%%%%%%%%%%%%%
% Short Sectioned Assignment
% LaTeX Template
% Version 1.0 (5/5/12)
%
% This template has been downloaded from:
% http://www.LaTeXTemplates.com
%
% Original author:
% Frits Wenneker (http://www.howtotex.com)
%
% License:
% CC BY-NC-SA 3.0 (http://creativecommons.org/licenses/by-nc-sa/3.0/)
%
%%%%%%%%%%%%%%%%%%%%%%%%%%%%%%%%%%%%%%%%%


%	PACKAGES AND OTHER DOCUMENT CONFIGURATIONS
%----------------------------------------------------------------------------------------

\documentclass[fontsize=11pt]{scrartcl} % A4 paper and 11pt font size

\usepackage[utf8]{inputenc}
\usepackage[T1]{fontenc} % Use 8-bit encoding that has 256 glyphs
\usepackage[english]{babel} % English language/hyphenation
\usepackage{amsmath,amsfonts,amsthm} % Math packages
\usepackage{sectsty} % Allows customizing section commands
\usepackage{mathtools}
\usepackage{graphicx}
\usepackage[all,cmtip]{xy}
\usepackage{centernot}
\usepackage{color}
\usepackage{bbm}
\usepackage{amssymb}
\usepackage{enumerate}
\usepackage{tcolorbox}
\usepackage{fancyhdr} % Custom headers and footers
\usepackage{geometry}
\usepackage{mathpazo}

\newif\ifsolutions
\solutionstrue

%\geometry{left=20mm,right=20mm,top=20mm}
%\geometry{a5paper, left=8mm,right=8mm,top=15mm,bottom=25mm}

%%% A5 paper size
% \geometry{papersize={168mm, 210mm}}% , left=20mm,right=20mm,top=30mm,bottom=40mm}
%%% Customized Paper size
%\geometry{papersize={180mm, 210mm}, left=28mm,right=28mm,top=15mm,bottom=25mm}

\newcommand{\cat}[1]{\mathbf{{#1}}}
\newcommand{\note}[1]{{\small \textit{\textbf{Note:} {#1}}}}
\def\C{\cat{C}}
\def\D{\cat{D}}
\def\E{\cat{E}}
\def\F{\cat{F}}
\def\Sets{\cat{Sets}}
\def\Grp{\cat{Grp}}
\def\Mon{\cat{Mon}}
\def\Ab{\cat{Ab}}
\def\Pos{\cat{Pos}}
\def\Cat{\cat{Cat}}
\def\Top{\cat{Top}}
\def\0{\cat{0}}
\def\1{\cat{1}}
\def\2{\cat{2}}
\def\3{\cat{3}}
\def\singleton{\{*\}}
\def\op{\textnormal{op}}



\def\H{\mathbb H}
\def\Q{\mathbb Q}
\def\R{\mathbb R}
\def\Z{\mathbb Z}
\def\Re{\textbf{Re}}
\def\Im{\textbf{Im}}
\def\id{\mathbbm{1}}

%%% ARROWS
\def\epi{\twoheadrightarrow}
\def\ipe{\twoheadleftarrow}
\def\mono{\rightarrowtail}
\def\onom{\rightarrowtail}
\def\xto{\xrightarrow}
\def\xinclusion{\xhookrightarrow}
\makeatletter
\providecommand*{\twoheadrightarrowfill@}{%
  \arrowfill@\relbar\relbar\twoheadrightarrow
}
\providecommand*{\twoheadleftarrowfill@}{%
  \arrowfill@\twoheadleftarrow\relbar\relbar
}
\providecommand*{\rightarrowtailfill@}{%
  \arrowfill@\relbar\relbar\rightarrowtail
}
\providecommand*{\leftarrowtailfill@}{%
  \arrowfill@\leftarrowtail\relbar\relbar
}
\newcommand*{\doublerightarrow}[2]{\mathrel{
  \settowidth{\@tempdima}{$\scriptstyle#1$}
  \settowidth{\@tempdimb}{$\scriptstyle#2$}
  \ifdim\@tempdimb>\@tempdima \@tempdima=\@tempdimb\fi
  \mathop{\vcenter{
    \offinterlineskip\ialign{\hbox to\dimexpr\@tempdima+1em{##}\cr
    \rightarrowfill\cr\noalign{\kern.5ex}
    \rightarrowfill\cr}}}\limits^{\!#1}_{\!#2}}}
\newcommand{\xepi}[2][]{\ext@arrow 0359\twoheadrightarrowfill@{#1}{#2}}
\newcommand{\xipe}[2][]{\ext@arrow 0359\twoheadleftarrowfill@{#1}{#2}}
\newcommand{\xmono}[2][]{\ext@arrow 0359\rightarrowtailfill@{#1}{#2}}
\newcommand{\xonom}[2][]{\ext@arrow 0359\leftarrowtailfill@{#1}{#2}}
\makeatother

%THEOREMS
\newenvironment{Solution}
  {\ifsolutions\begin{tcolorbox}[colback=blue!8!white,colframe=blue!35!black,title=Solution]}
  {\end{tcolorbox}\fi}


\newcounter{booksection}
\theoremstyle{definition}
\newtheorem{problem}{Problem}
\newtheorem{bookproblem}{Problem}[booksection]
\newtheorem*{lemmma}{Lemma}
\newtheorem*{solution}{Solution}
\newtheorem*{fact}{Fact}
\newtheorem*{definition}{Definition}
\newtheorem*{definitions}{Definitions}
\theoremstyle{theorem}
\newtheorem{corollary}{Corollary}
\newtheorem{theorem}{Theorem}
\newcommand{\irr}{\text{irr}}
\newcommand{\ol}[1]{\overline{#1}}
%NEW COMMANDS
\newcommand{\mc}[1]{\mathcal{#1}}
\newcommand{\bb}[1]{\mathbb{#1}}
\newcommand{\bbm}[1]{\mathbbm{#1}}
\newcommand{\ms}[1]{\mathscr{#1}}
\newcommand{\ttt}[1]{\texttt{#1}}
\newcommand{\includecode}[2][Python]{\lstinputlisting[caption=#2, escapechar=, style=custom#1]{#2}}
\newcommand{\embf}[1]{\textbf{\emph{#1}}}
\newcommand{\tbf}[1]{\textbf{#1}}
	% TOPOLOGY COMMANDS
\newcommand{\intr}[1]{\accentset{\circ}{#1}}
\newcommand{\bndr}{\partial}
\newcommand{\clsr}[1]{\overline{#1}}
\newcommand{\tpl}[1][]{\mathscr{T}_{#1}}
	% COMPLEX VARIABLE COMMANDS
\newcommand{\Log}{\text{Log}}
\newcommand{\partials}[2]{\frac{\partial #1}{\partial #2}}
\newcommand{\hessian}[3]{\left(\begin{array}{cc}\frac{\partial^2 #1}{\partial #2^2} & \frac{\partial^2 #1}{\partial #1 \partial #2} \\ \frac{\partial^2 u}{\partial #1\partial #2} & \frac{\partial^2 #1}{\partial #2^2}\end{array}\right)}
\newcommand{\Res}{\text{Res}}
    % LINEAR ALGEBRA COMMANDS
\newcommand{\Span}{\text{span}}
\newcommand{\Null}{\text{Null}}
\newcommand{\Rank}{\text{rank}}
\newcommand{\Mat}[2]{\,\text{Mat}_{{#1}\times{#2}}}
\renewcommand{\u}{\vec u}
\renewcommand{\v}{\vec v}
\newcommand{\w}{\vec w}
\newcommand{\x}{\vec x}
\newcommand{\y}{\vec y}
\newcommand{\z}{\vec z}
\newcommand{\im}{\text{im}}
    % ALGEBRA COMMANDS
\newcommand{\Stab}[2]{\,\text{Stab}_{#1}({#2})}
\newcommand{\Cent}[2]{\,\text{C}_{#1}({#2})}
\newcommand{\Center}[1]{\,\text{Z}({#1})}
\newcommand{\Norm}[2]{\,\text{N}_{#1}({#2})}
\newcommand{\subgp}{\leq}
\newcommand{\Orb}[1]{\,{#1}\text{-orbit}}
\newcommand{\orbit}[2]{\,\text{orbit}_{#1}({#2})}
\newcommand{\Inn}[1]{\,\text{Inn}({#1})}
\newcommand{\Aut}[1]{\,\text{Aut}({#1})}
\newcommand{\Syl}{\,\text{Syl}}
\newcommand{\normal}{\trianglelefteq}
    % OTHER
\newcommand{\owl}{\widehat{\dbinom{\odot_\text{v}\odot}{\wr}}}

\allsectionsfont{\centering \normalfont\scshape} % Make all sections centered, the default font and small caps

\pagestyle{fancyplain} % Makes all pages in the document conform to the custom headers and footers
\fancyhead{} % No page header - if you want one, create it in the same way as the footers below
\fancyfoot[L]{} % Empty left footer
\fancyfoot[C]{} % Empty center footer
\fancyfoot[R]{\thepage} % Page numbering for right footer
\renewcommand{\headrulewidth}{0pt} % Remove header underlines
\renewcommand{\footrulewidth}{0pt} % Remove footer underlines
\setlength{\headheight}{13.6pt} % Customize the height of the header

\numberwithin{equation}{section} % Number equations within sections (i.e. 1.1, 1.2, 2.1, 2.2 instead of 1, 2, 3, 4)
\numberwithin{figure}{section} % Number figures within sections (i.e. 1.1, 1.2, 2.1, 2.2 instead of 1, 2, 3, 4)
\numberwithin{table}{section} % Number tables within sections (i.e. 1.1, 1.2, 2.1, 2.2 instead of 1, 2, 3, 4)

\setlength\parindent{0pt} % Removes all indentation from paragraphs - comment this line for an assignment with lots of text
\newcommand{\horrule}[1]{\rule{\linewidth}{#1}} % Create horizontal rule command with 1 argument of height



\title{\normalfont\normalsize 
\textsc{Umass Amherst} \\ [10pt]% Your university, school and/or department name(s)
\horrule{0.5pt} \\[0.4cm] % Thin top horizontal rule
\huge Fun With Cats --- Homework 4 \\
\normalsize Epis, Monos, and Abstract Structures
\horrule{2pt} \\[-0.8cm] % Thick bottom horizontal rule
}
\author{}
\date{\normalsize\today}
\begin{document}
\maketitle

\section{Exercises}

\setcounter{booksection}{2}
\showsolutionstrue
%%% 2.13
\begin{BookProblem}{13}
  In any category with binary products, show directly that \[A \times (B \times
  C) \cong (A \times B) \times C \]

  \begin{Solution}
    This follows from the UMPs of \(A\times (B \times C)\), \((A \times B)\times
    C\), and \(A \times B \times C\). Note that there exists a unique \(u\)
    making the following (rather ugly) diagram commute:


    \begin{equation}
      \def\piA{\ar[ddl]_{\pi_A}}
      \def\piBC{\ar[dr]^{\pi_{B\times C}}}
      \def\piB{\ar@/^4.2pc/[ddl]^{\pi_B}}
      \def\piC{\ar[d]_{\pi_C}}
      \def\pA{\ar[l]_{p_A}}
      \def\pB{\ar[d]_{p_B}}
      \def\pC{\ar[r]^{p_C}}
      \def\ps{\pA\pB\pC}
      \def\induced{\ar@{..>}[dd]^u}
      \xymatrix{%
        & A \times (B \times C)\piA\piBC\induced &  \\
          &                                & B \times C\piB\piC\\
        A & A \times B \times C\ps         & C \\
          &          B                     &
      }
    \end{equation}

    Likewise we have a unique arrow \(u'\) making the following (even uglier)
    diagram commute.

    \begin{equation}
      \def\pC{\ar[ddr]^{p_C}}
      \def\pAB{\ar[dl]_{p_A \times p_B}}
      \def\ppA{\ar@/_4.2pc/[ddr]^{p'_A}}
      \def\ppB{\ar[d]^{p'_B}}
      \def\piAB{\ar[ul]_{\pi_{A\times B}}}
      \def\piC{\ar[r]^{\pi_C}}
      \def\pis{\piAB\piC}
      \def\induced{\ar@{..>}[dd]^{u'}}
      \xymatrix{%
                            & A \times B \times C\pAB\pC\induced     &  \\
        A \times B \ppA\ppB &                                & \\
        A                   & (A \times B) \times C\pis      & C \\
                            &          B                     &
      }
    \end{equation}

    We have a unique
  \end{Solution}
\end{BookProblem}

\clearpage
%%% 2.14
\begin{BookProblem}{14}
  \begin{enumerate}[(a)]
    \item For any index set \(I\), define the product \(\prod_{i \in I}X_i\) of
      an \(I\)-indexed family of objects \((X_i)_{i \in I}\) in a category, by
      giving a UMP generalizing that for binary products (the case \(I = 2\)).
    \item Show that in \(\Sets\), for any set \(X\) the set \(X^I\) of all
      functions \(f : I \to X\) has the UMP, with respect to the ``constant
      family'' where \(X_i = X\) for all \(i \in I\), and thus

      \[X^I \cong \prod_{i \in I}X.\]
  \end{enumerate}
  \begin{Solution}
    \begin{enumerate}[(a)]
      %%% SOLUTION FOR (A)
      \item The product \(\prod_{i \in I} X_i\) is a collection of diagrams
        \begin{equation}
          \Pi_I \xto{\pi_i}X_i,\quad i \in I
        \end{equation}
        such that for any other collection of diagrams
        \begin{equation}
          T \xto{f_i}X_i,\quad i \in I
        \end{equation}
        there is a unique \(u : T \to \Pi_I\) making all induced subdiagrams
        commute:
        \begin{equation}
          \xymatrix{
            X\ar[d]^u\ar[dr]^{f_i} & \\
            \Pi_I\ar[r]_{\pi_i} & X_i
          }
        \end{equation}

      %%% SOLUTION FOR (B)
      \item Consider sets \(X\) and \(I\), and consider the set of functions
        \(X^I\) of all functions \(I \to X\). We set up our diagram
        \begin{equation}
          %\xymatrix{ }
        \end{equation}
    \end{enumerate}

  \end{Solution}
\end{BookProblem}

\clearpage
%%% 2.17
\begin{BookProblem}{17}
  In any category \(\C\) with products, define the \textbf{graph} of an arrow
  \(f : A \to B\) to be the monomorphism
  \[\Gamma(f) = \langle 1_A, f\rangle : A \mono A \times B\]
  (Why is this monic?). Show that for \(\C = \Sets\) this determines a functor
  \(\Gamma : \Sets \to \Rel\) to the category \(\Rel\) of relations, as defined
  in the exercies to Chapter 1. (To get an actual relation \(R(f) \subset A
  \times B\), take the image of \(\Gamma(f) : A \mono A \times B\).)
  \begin{Solution}
    To show that \(\Gamma(f)\) is monic, consider \(X \doublerightarrow{a}{b} A
    \xto{\Gamma(f)} A \times B \xto{p_A} A\), with \(p_A\) the projection
    map\ldots\\

    Now we wish to show that \(\Gamma(f)\) determines a functor from \(\Sets \to
    \Rel\). Recall that the objects of \(\Rel\) are sets and the arrows \(A \to
    B\) are binary relations \(R \subseteq A \times B\). Take the object map
    \(\Gamma_\obj\) to map sets to themselves, and define the hom map by
    \(\Gamma_{\hom} (f: A \to B) = \Gamma(f)(A) = \{(a, f(a)) : a \in A\}\).\\

    Identity maps go to identity relations, and composition works as expected:
    for \(g : B \to C\) we have \(\Gamma(g\circ f)(A) = \{(a, g(f(a))) : a \in
    A\} = \{(g, g(b)) : b \in B\} \circ_\Rel \{(a, f(a)) : a \in A\} = \Gamma(g)
    \circ_\Rel \Gamma(f)\).

  \end{Solution}
\end{BookProblem}

\clearpage
%%% 2.18
\begin{BookProblem}{18}
  Show that the forgetful functor \(U : \Mon \to \Sets\) from monoids to sets is
  representable. Infer that \(U\) preserves all (small) products.
  \begin{Solution}
    \note{I haven't checked this proof too carefuly and there might be some
    subtleties I didn't notice---take with a grain of salt. In particular, this
    may not hold for general representable functors and we need to bring in some
    domain specific properties of \(\Mon\) or \(\Sets\) to complete the proof.
    Even so, if such corrections need be made we should be able to tweak the
    following proof appropriately.}\\

    Consider the functor \(F = \hom(M(1), -)\). For a monoid \(N\) there is
    precisely one monoid homomorphism \(\varphi_x : M(1) \to N\) mapping \(1
    \mapsto x\) for each \(x \in N\); that is, there is a bijection between
    the image of \(\hom(M(1), N)\) and the underlying set \(|N| = U(N)\), call
    it \(\eta : F(N)(N) \cong |N|\). Then \(\eta \circ F = U\) and \(U\)
    is representable (up to an iso).\\

    To prove that \(U\) preserves small products we update the function
    \(\varphi_X\) defined in \S2.7 to operate on small product diagrams: for object
    \(P\) and \(I\)-indexed family of objects \({(A_i)}_{i \in I}\) with arrows
    \({(p_i : P \to A_i)}_{i \in I}\), and for object \(X\), define 
    %
    \[\varphi_X : \hom(X, P) \to \prod_{i \in I} \hom(X, A_i)\]
    %
    by
    %
    \[\varphi_X(x:X \to P) = {(p_i \circ x)}_{i \in I}.\]

    Then the proof of Proposition 2.20 is the same, and Corollary 2.22 follows;
    in particular, representable functors preserve small products.\qed{}
  \end{Solution}
\end{BookProblem}


%%%%%%%%%%%%%%%%%%%%%%%%%%%%%%%%%%%%%%%%%%%%%%%%%%%%%%%%%%%%%%%%%%%%%%%%%%%%%%%%
%%%%%%%%%%%%%%%%%%%%%%%%%%%%%%%%%% Chapter 3 %%%%%%%%%%%%%%%%%%%%%%%%%%%%%%%%%%%
%%%%%%%%%%%%%%%%%%%%%%%%%%%%%%%%%%%%%%%%%%%%%%%%%%%%%%%%%%%%%%%%%%%%%%%%%%%%%%%%
\setcounter{booksection}{3}
\clearpage
%%% 3.1
\begin{BookProblem}{1}
  In any category \(\C\), show that 
  \[\xymatrix{A\ar[rr]_{c_1}&&C&&B\ar[ll]^{c_2}}\]
  is a coproduct diagram just if for every object \(Z\), the map 
  \begin{align*}
    \hom(C,Z) &\to \hom(A,Z) \times \hom(B,Z)\\
    f &\mapsto \langle f \circ c_1, f \circ c_2 \rangle
  \end{align*}
  is an iso. Do this by using duality, taking the corresponding fact about
  products as given.

  \begin{Solution}
    From \S2.7 we have the fact that 
    \[\xymatrix{A\ar@{<-}[rr]_{p_1}&&C&&B\ar@{<-}[ll]^{p_2}}\]
    is a product diagram precisely when for every object \(X\) the map
    \begin{align}
      \vartheta_X: \hom(X,P) &\to     \hom(X, A) \times \hom(X, B)\label{eq}\\
                   x         &\mapsto [p_1\circ x, p_2 \circ x]\nonumber
    \end{align}
    is an iso. That is, for any diagram \(A \xleftarrow{a} Y\xrightarrow{b} B\)
    there is a unique \(u: Y \to C\) making the induced diagrams commute
    precisely when \(\vartheta_X\) is an iso:

    \[\xymatrix{
      && Y\ar[ddll]_a\ar[ddrr]^b\ar@{..>}[dd]^u\\
      &&&&\\
      A\ar@{<-}[rr]_{p_1}&&C&&B\ar@{<-}[ll]^{p_2}
    }\]
    
    Now that we have stated our problem in purely categorical terms we may apply
    duality and get our result for free.\qed{}
  \end{Solution}
\end{BookProblem}

\clearpage
%%% 3.2
\begin{BookProblem}{2}
  Show in detail that the free monoid functor \(M\) preserves coproducts: for any
  set \(A\), \(B\),
  \[M(A) + M(B) \cong M(A + B)\quad\text{(canonically).}\]
  Do this as indicated in the text by using the UMPs of the coproducts \(A + B\)
  and \(M(A) + M(B)\) and of free monoids.\\

  \note{We did this in class but it is worth working through the proof.}
  \begin{Solution}
    We begin by duplicating the free-monoid diagram from \S3.1, adding labels to
    help us along (we will define these as we go).
    \begin{equation}
      \xymatrix{
      & N & \\
      %&&&&\\
      M(A)\ar[r]^{j_A}\ar[ur]^f & M(A + B)\ar@{..>}[u] & M(B)\ar[l]_{j_B}\ar[lu]_g\\
      %&&&&\\
      A\ar[r]_{i_A}\ar[u]^{\eta_A} & A + B\ar[u]_{\eta_{A+B}} & B\ar[l]^{i_B}\ar[u]_{\eta_B}
    }
    \end{equation}

    The bottom row comes from the UMP of \(A + B\) in \(\Sets\), and from the
    UMP of \(M(X)\) we get maps \(\eta_*\) into the associated free monoids.
    Next, define \(j_A\) and \(j_B\) to be the maps induced by the UMP of
    \(M(A)\) and \(M(B)\) respectively:
    \begin{equation}
      \xymatrix{
        M(A)\ar@{..>}[rr]^{j_A} && M(A + B)\\
        &&\\
        A\ar[uurr]_{\ \ {i_A}\circ \eta_{A + B}}\ar[uu]^{\eta_A} &&
      }
    \end{equation}

    Now we want to show that \(M(A) \xlongrightarrow{j_A} M(A + B)
    \xlongleftarrow{j_B} M(B)\) has the UMP of \(M(A) + M(B)\). To this end,
    take diagram
    \[ M(A) \xlongrightarrow{f} N \xlongleftarrow{g} M(B). \]

    We may define function \(\varphi : A + B\) to \(N\) by

    \[\varphi(x) = \begin{dcases*}
      (f \circ\eta_A)(x) & if \(x \in A\)\\
      (g \circ\eta_B)(x) & if \(x \in B\)
    \end{dcases*}
    \]
    Apply the UMP of \(M(A + B)\) and we win.\qed{}
  \end{Solution}
\end{BookProblem}

%%% 3.3
\begin{BookProblem}{3}
  Verify that the construction given in the text of the coproduct of monoids \(A
  + B\) as a quotient of the free monoid \(M(|A| + |B|)\) really is a coproduct
  in the category of monoids.
  \begin{Solution}
  \end{Solution}
\end{BookProblem}

%%% 3.4
\begin{BookProblem}{4}
  Show that the product of two powerset Boolean algebras \(\mathcal{P}(A)\) and
  \(\mathcal{P}(B)\) is also a powerset, namely of the coproduct of the sets
  \(A\) and \(B\),
  \[\mathcal{P}(A) \times \mathcal{P}(B) \cong \mathcal{P}(A + B).\]
  (Hint: determine the projections \(\pi_1 : \mathcal{P}(A + B) \to
  \mathcal{P}(A)\) and \(\pi_2: \mathcal{P}(A + B) \to \mathcal{P}(B)\), and
  check that they have the UMP of the product.)
  \begin{Solution}
  \end{Solution}
\end{BookProblem}

\section{Additional Exercises}
%%% 2.16
\begin{problem}[Awodey, 2.16]
  In the category of types \(\C(\lambda)\) of the \(\lambda\)-calculus,
  determine the product functor \(A, B \mapsto A \times B\) explicitly. Also
  show that for any fixed type \(A\) there is a functor \(A \to (-) :
  \C(\lambda) \to \C(\lambda)\), taking any type \(X\) to \(A \to X\).
\end{problem}

%%% 3.5
\begin{problem}[Awodey, 3.5]
  Consider the category of proofs of a natural deduction system with disjunction
  introduction and elimination rules. Identify proofs under the equations
  \[[p,q] \circ i_1 = p,\quad\quad [p,q] \circ i_2 = q\]
  \[[r \circ i_1, r \circ i_2] = r\]
  for any \(p : A \to C, q : B \to C\), and \(r : A + B \to C\). By passing to
  equivalence classes of proofs awith respect to the equivalence relation
  generated by these equations (i.e., two proofs are equivalent if you can get
  one from the other by removing all such ``detours''). Show that the resulting
  category does indeed have coproducts.
  \begin{Solution}
  \end{Solution}
\end{problem}

\end{document}
