%%%%%%%%%%%%%%%%%%%%%%%%%%%%%%%%%%%%%%%%%
% Short Sectioned Assignment
% LaTeX Template
% Version 1.0 (5/5/12)
%
% This template has been downloaded from:
% http://www.LaTeXTemplates.com
%
% Original author:
% Frits Wenneker (http://www.howtotex.com)
%
% License:
% CC BY-NC-SA 3.0 (http://creativecommons.org/licenses/by-nc-sa/3.0/)
%
%%%%%%%%%%%%%%%%%%%%%%%%%%%%%%%%%%%%%%%%%

%----------------------------------------------------------------------------------------
%	PACKAGES AND OTHER DOCUMENT CONFIGURATIONS
%----------------------------------------------------------------------------------------

\documentclass[fontsize=12pt]{scrartcl} % A4 paper and 11pt font size

\usepackage[utf8]{inputenc}
\usepackage[T1]{fontenc} % Use 8-bit encoding that has 256 glyphs
\usepackage[english]{babel} % English language/hyphenation
\usepackage{amsmath,amsfonts,amsthm} % Math packages
\usepackage{sectsty} % Allows customizing section commands
\usepackage{mathtools}
\usepackage{centernot}
\usepackage[all,cmtip]{xy}
\usepackage{color}
\usepackage{bbm}
\usepackage{amssymb}
\usepackage{enumerate}
\usepackage{fancyhdr} % Custom headers and footers
\usepackage{geometry}
%\usepackage{mathpazo}

%\geometry{left=20mm,right=20mm,top=20mm}

\def\R{\mathbb R}
\def\RS{\mathbb S}
\def\F{\mathbb F}
\def\Z{\mathbb Z}
\def\Q{\mathbb Q}
\def\C{\mathbb C}
\def\D{\mathbb D}
\def\H{\mathbb H}
\def\Re{\textbf{Re}}
\def\Im{\textbf{Im}}
\def\1{\mathbbm{1}}

%THEOREMS
\theoremstyle{definition}
\newtheorem{prb}{Problem}
\newtheorem*{prob}{\textbf{Problem :}}
\newtheorem*{prf}{Proof}
\newtheorem*{lem}{Lemma}
\newtheorem*{sln}{Solution}
\newtheorem*{fct}{Fact}
\newtheorem*{dfn}{Definition}
\newtheorem*{dfns}{Definitions}
\theoremstyle{theorem}
\newtheorem{cor}{Corollary}
\newtheorem{thm}{Theorem}
\newcommand{\irr}{\text{irr}}
\newcommand{\ol}[1]{\overline{#1}}
%NEW COMMANDS
\newcommand{\mc}[1]{\mathcal{#1}}
\newcommand{\bb}[1]{\mathbb{#1}}
\newcommand{\bbm}[1]{\mathbbm{#1}}
\newcommand{\ms}[1]{\mathscr{#1}}
\newcommand{\ttt}[1]{\texttt{#1}}
\newcommand{\includecode}[2][Python]{\lstinputlisting[caption=#2, escapechar=, style=custom#1]{#2}}
\newcommand{\embf}[1]{\textbf{\emph{#1}}}
\newcommand{\tbf}[1]{\textbf{#1}}
	% TOPOLOGY COMMANDS
\newcommand{\intr}[1]{\accentset{\circ}{#1}}
\newcommand{\bndr}{\partial}
\newcommand{\clsr}[1]{\overline{#1}}
\newcommand{\tpl}[1][]{\mathscr{T}_{#1}}
	% COMPLEX VARIABLE COMMANDS
\newcommand{\Log}{\text{Log}}
\newcommand{\partials}[2]{\frac{\partial #1}{\partial #2}}
\newcommand{\hessian}[3]{\left(\begin{array}{cc}\frac{\partial^2 #1}{\partial #2^2} & \frac{\partial^2 #1}{\partial #1 \partial #2} \\ \frac{\partial^2 u}{\partial #1\partial #2} & \frac{\partial^2 #1}{\partial #2^2}\end{array}\right)}
\newcommand{\Res}{\text{Res}}
    % LINEAR ALGEBRA COMMANDS
\newcommand{\Span}{\text{span}}
\newcommand{\Null}{\text{Null}}
\newcommand{\Rank}{\text{rank}}
\newcommand{\Mat}[2]{\,\text{Mat}_{{#1}\times{#2}}}
\renewcommand{\u}{\vec u}
\renewcommand{\v}{\vec v}
\newcommand{\w}{\vec w}
\newcommand{\x}{\vec x}
\newcommand{\y}{\vec y}
\newcommand{\z}{\vec z}
\newcommand{\im}{\text{im}}
    % ALGEBRA COMMANDS
\newcommand{\Stab}[2]{\,\text{Stab}_{#1}({#2})}
\newcommand{\Cent}[2]{\,\text{C}_{#1}({#2})}
\newcommand{\Center}[1]{\,\text{Z}({#1})}
\newcommand{\Norm}[2]{\,\text{N}_{#1}({#2})}
\newcommand{\subgp}{\leq}
\newcommand{\Orb}[1]{\,{#1}\text{-orbit}}
\newcommand{\orbit}[2]{\,\text{orbit}_{#1}({#2})}
\newcommand{\Inn}[1]{\,\text{Inn}({#1})}
\newcommand{\Aut}[1]{\,\text{Aut}({#1})}
\newcommand{\Syl}{\,\text{Syl}}
\newcommand{\normal}{\trianglelefteq}
    % OTHER
\newcommand{\owl}{\widehat{\dbinom{\odot_\text{v}\odot}{\wr}}}

\allsectionsfont{\centering \normalfont\scshape} % Make all sections centered, the default font and small caps

\pagestyle{fancyplain} % Makes all pages in the document conform to the custom headers and footers
\fancyhead{} % No page header - if you want one, create it in the same way as the footers below
\fancyfoot[L]{} % Empty left footer
\fancyfoot[C]{} % Empty center footer
\fancyfoot[R]{\thepage} % Page numbering for right footer
\renewcommand{\headrulewidth}{0pt} % Remove header underlines
\renewcommand{\footrulewidth}{0pt} % Remove footer underlines
\setlength{\headheight}{13.6pt} % Customize the height of the header

\numberwithin{equation}{section} % Number equations within sections (i.e. 1.1, 1.2, 2.1, 2.2 instead of 1, 2, 3, 4)
\numberwithin{figure}{section} % Number figures within sections (i.e. 1.1, 1.2, 2.1, 2.2 instead of 1, 2, 3, 4)
\numberwithin{table}{section} % Number tables within sections (i.e. 1.1, 1.2, 2.1, 2.2 instead of 1, 2, 3, 4)

\setlength\parindent{0pt} % Removes all indentation from paragraphs - comment this line for an assignment with lots of text
\newcommand{\horrule}[1]{\rule{\linewidth}{#1}} % Create horizontal rule command with 1 argument of height



\title{\normalfont\normalsize 
\textsc{Umass Amherst} \\ [10pt]% Your university, school and/or department name(s)
\horrule{0.5pt} \\[0.4cm] % Thin top horizontal rule
\huge Fun With Cats --- Homework 2 \\ % The assignment title
\horrule{2pt} \\[-0.8cm] % Thick bottom horizontal rule
}
\author{} % Your name
\date{\normalsize\today} % Today's date or a custom date
\begin{document}
\maketitle % Print the title
Hi all, here are problems for this week. I've typed them up since I know that
not everyone has the current text, and I also wanted to add a couple extra
problems from Mac Lane as `extra credit'.\\

The first few problems are on the easier side, mainly ensuring that you
understand definitions. Problems 11 and 12 make you work with UMPs which is much
more useful than watching someone prattle on about them. I've also included
three problems at the end which are a bit harder but should be doable. Problem 2
in particular is a good one as this has a very nice `algebraic' solution that
comes up in a number of contexts, including a proof of Brouwer's Fixed Point
Theorem. The last of the Mac Lane problems is fairly simple---there's really
only one thing you can do.

\section*{Chapter 1: Categories}
\setcounter{section}{1}

\setcounter{problem}{4}
\def\dom{\cat{dom}}
\def\cod{\cat{cod}}

%%% PROBLEM 5
\begin{problem}
  For any category \(\C\), define a functor \(U:\C/C \to \C\) from the slice
  category over an object \(C\) that ``forgets about C.'' Find a functor \(F :
  \C / C \to \C^\to\)  to the arrow category such that \(\dom \circ F =
  U\).\footnote{Recall that \(\C^\to\) is equipped with two functors
  \begin{equation*}
    \xymatrix{\C & \C^\to\ar[l]_\dom \ar[r]^\cod & \C}.
  \end{equation*}
  }\qed{}
\end{problem}

%%% PROBLEM 6
\begin{problem} 
  Construct the ``coslice category'' \(C/\C\) of a category \(\C\) under an
  object \(C\) from the slice category \(\C/C\) and the ``dual category''
  operation \(-^\op\).
\end{problem}

%%% PROBLEM 7
\begin{problem}
  Let \(\{a,b\}\) be any set with exactly 2 elements \(a\) and \(b\). Define a
  functor \(F:\Sets / 2 \to \Sets \times \Sets\) with \(F(f : X \to 2) =
  (f^{-1}(a), f^{-1}(b))\). Is this an isomorphism of categories? What about the
  analogous situation with a one-lement set \(1 = \{a\}\) instead of
  \(2\)?\qed{}
\end{problem}

%%% PROBLEM 9
\setcounter{problem}{8}
\begin{problem}
  Describe the free categories on the following graphs by determining their
  objects, arrows, and composition operations.
  \begin{enumerate}[(a)]
    \item
      \begin{equation*}
        \xymatrix{ a \ar[rr]^e && b }
      \end{equation*}

    \item
      \begin{equation*}
        \xymatrix{a\ar@<0.5ex>[rr]^e && b\ar@<0.5ex>[ll]^f}
      \end{equation*}

    \item 
      \begin{equation*}
        \xymatrix{a \ar[rr]^e\ar[ddrr]_g && b\ar[dd]^f \\ && \\ && c}
      \end{equation*}

    \item 
      \begin{equation*}
        \xymatrix{
          a \ar@<0.5ex>[rr]^e && b\ar@<0.5ex>[ll]^h\ar[dd]^f & d\\ 
          && &\\
                              && c \ar[uull]^g & }
      \end{equation*}
  \end{enumerate}\qed{}
\end{problem}


%%% PROBLEM 10
\begin{problem}
  How may free categories on graphs are there which have exactly six arrows?
  Draw the graphs that generate these categories.\qed{}
\end{problem}


%%% PROBLEM 11
\begin{problem}
  Show that the free monoid functor
  \begin{equation*}
    M : \Sets \to \Mon
  \end{equation*}

  exists in two different ways:
  \begin{enumerate}[(a)]
    \item Assume the particular choice \(M(X) = X^*\) and define its effect
      \[ M(f) : M(A) \to M(B)\]
      on a function \(f : A \to B\) to be
      \[M(f)(a_1 \cdots a_k) = f(a_1) \cdots f(a_k), a_i \in A.\]
    \item Assume only the UMP of the free monoid and use it to determine \(M\)
      on functions, showing the result to be a functor.
  \end{enumerate}

  Reflect on how these two approaches are related.\qed{}
\end{problem}

%%% PROBLEM 12
\begin{problem}
  Verify the UMP for free categories on graphs. Specifically, let \(\C(G)\) be
  the free category on the graph \(G\) and \(i : G \to U(\C(G))\) be the
  `inclusion' graph homomorphism taking vertices and edges to themselves. Show
  that for any category \(\D\) and graph homomorphism \(f : G \to U(\D)\), there
  is a unique functor 
  \[\bar h : \C(G) \to \D\] 
  with 
  \[U(\bar h) \circ i = h.\]

\textbf{Note:} refer to the UMP for all the movers and shakers in the above
  problem.\qed{}
\end{problem}

\section*{Some (Slightly Adapted) Problems from Mac Lane}

Note that these are more difficult (and thus more fun).

\setcounter{problem}{0}
\begin{problem}
  Recall the following categories:
  \begin{itemize}
    \item \(\1\):\begin{equation*}
        *
    \end{equation*}
  \item \(\2\): \begin{equation*}
      \xymatrix{\ast\ar[r] & \star}
    \end{equation*}
  \item \(\3\): \begin{equation*}
      \xymatrix{\ast\ar[r]\ar[rd] & \star\ar[d]\\&\circ}
    \end{equation*}
  \end{itemize}

  Given a category \(\C\), describe the functors of each of the following forms:
  \begin{itemize}
    \item \(\1 \to \C\)
    \item \(\2 \to \C\)
    \item \(\3 \to \C\)
  \end{itemize}

  What is a nice way to characterize these functors? That is, how would you
  characterize the class of functors \(\1 \to \C\)? \(\2 \to \C\)? \(\3 \to
  \C\)?\qed{}
\end{problem}

\begin{problem} Recall that a group \(G\) is \textit{abelian} if \(ab = ba\) for
  all \(a,b \in G\). Let \(\Ab\) be the category of abelian groups and group
  homomorphisms and \(\Grp\) be the category of groups and group
  homomorphisms.
  For a group \(G\) define the \textit{center} of \(G\), denoted \(Z(G)\), to be
  the set of all \(x \in G\) that commutes with all of \(G\); that is,
  \[Z(G) = \{x \in G : \forall y \in G xy = yx\}.\]
  It is a fact that \(Z(G)\) is a subgroup of \(G\) \textit{(proof!)}.
  Show that there is no functor sending each group \(G\) to its center.
  (\textbf{hint:} recall that \(S_n\) is permutation group on \(n\) elements,
  called the symmetric group. Consider \(S_2 \to S_3 \to S_2\)).\\

  \textbf{Note:} in the statement of the problem in Mac Lane (which is much
  shorter and has none of the background I give) it is a bit ambiguous what is
  meant by ``send each group to its center''. However there is only one
  reasonable interpretation of this functor: this \(F\) would map each group
  \(G\) to its center, taking each element \(z \in Z(G)\) to itself. In
  mathematics we call this a retract of \(G\) onto \(Z(G)\): that is, \(F\circ F
  = F\), this term being rooted in topology (as far as I know). We will see more
  of retractions in section 2.1 next Thursday. If I recall I'll bring in my copy
  of Hatcher's Algebraic Topology as this gives motivation for the name (much of
  the motivation for the subject comes from topology so I will continue to
  reference it).\\

  Speaking of topology, this problem above has a very similar flavor to (one of)
  the (many) proof(s) of the following fixed point theorem: 
  \begin{theorem}[Brouwer]
    Every continuous function from a disk to itself has a fixed point.
  \end{theorem}
  We won't prove this since it involves some background in fundamental groups
  but the punchline of the proof is the same as the proof here.\qed{}

\end{problem}

\begin{problem}
  Find two different functors \(F: \Grp \to \Grp\) such that the object map
  \(F(G) = G\) is the identify.\qed{}
\end{problem}

\end{document}
