
%	PACKAGES AND OTHER DOCUMENT CONFIGURATIONS
%----------------------------------------------------------------------------------------

\documentclass[fontsize=11pt]{scrartcl} % A4 paper and 11pt font size

\usepackage[utf8]{inputenc}
\usepackage[T1]{fontenc} % Use 8-bit encoding that has 256 glyphs
\usepackage[english]{babel} % English language/hyphenation
\usepackage{amsmath,amsfonts,amsthm} % Math packages
\usepackage{sectsty} % Allows customizing section commands
\usepackage{mathtools}
\usepackage{graphicx}
\usepackage[all,cmtip]{xy}
\usepackage{centernot}
\usepackage{color}
\usepackage{bbm}
\usepackage{amssymb}
\usepackage{enumerate}
\usepackage{tcolorbox}
\usepackage{fancyhdr} % Custom headers and footers
\usepackage{geometry}
\usepackage{mathpazo}

\newif\ifsolutions
\solutionstrue

%\geometry{left=20mm,right=20mm,top=20mm}
%\geometry{a5paper, left=8mm,right=8mm,top=15mm,bottom=25mm}

%%% A5 paper size
% \geometry{papersize={168mm, 210mm}}% , left=20mm,right=20mm,top=30mm,bottom=40mm}
%%% Customized Paper size
%\geometry{papersize={180mm, 210mm}, left=28mm,right=28mm,top=15mm,bottom=25mm}

\newcommand{\cat}[1]{\mathbf{{#1}}}
\newcommand{\note}[1]{{\small \textit{\textbf{Note:} {#1}}}}
\def\C{\cat{C}}
\def\D{\cat{D}}
\def\E{\cat{E}}
\def\F{\cat{F}}
\def\Sets{\cat{Sets}}
\def\Grp{\cat{Grp}}
\def\Mon{\cat{Mon}}
\def\Ab{\cat{Ab}}
\def\Pos{\cat{Pos}}
\def\Cat{\cat{Cat}}
\def\Top{\cat{Top}}
\def\0{\cat{0}}
\def\1{\cat{1}}
\def\2{\cat{2}}
\def\3{\cat{3}}
\def\singleton{\{*\}}
\def\op{\textnormal{op}}



\def\H{\mathbb H}
\def\Q{\mathbb Q}
\def\R{\mathbb R}
\def\Z{\mathbb Z}
\def\Re{\textbf{Re}}
\def\Im{\textbf{Im}}
\def\id{\mathbbm{1}}

%%% ARROWS
\def\epi{\twoheadrightarrow}
\def\ipe{\twoheadleftarrow}
\def\mono{\rightarrowtail}
\def\onom{\rightarrowtail}
\def\xto{\xrightarrow}
\def\xinclusion{\xhookrightarrow}
\makeatletter
\providecommand*{\twoheadrightarrowfill@}{%
  \arrowfill@\relbar\relbar\twoheadrightarrow
}
\providecommand*{\twoheadleftarrowfill@}{%
  \arrowfill@\twoheadleftarrow\relbar\relbar
}
\providecommand*{\rightarrowtailfill@}{%
  \arrowfill@\relbar\relbar\rightarrowtail
}
\providecommand*{\leftarrowtailfill@}{%
  \arrowfill@\leftarrowtail\relbar\relbar
}
\newcommand*{\doublerightarrow}[2]{\mathrel{
  \settowidth{\@tempdima}{$\scriptstyle#1$}
  \settowidth{\@tempdimb}{$\scriptstyle#2$}
  \ifdim\@tempdimb>\@tempdima \@tempdima=\@tempdimb\fi
  \mathop{\vcenter{
    \offinterlineskip\ialign{\hbox to\dimexpr\@tempdima+1em{##}\cr
    \rightarrowfill\cr\noalign{\kern.5ex}
    \rightarrowfill\cr}}}\limits^{\!#1}_{\!#2}}}
\newcommand{\xepi}[2][]{\ext@arrow 0359\twoheadrightarrowfill@{#1}{#2}}
\newcommand{\xipe}[2][]{\ext@arrow 0359\twoheadleftarrowfill@{#1}{#2}}
\newcommand{\xmono}[2][]{\ext@arrow 0359\rightarrowtailfill@{#1}{#2}}
\newcommand{\xonom}[2][]{\ext@arrow 0359\leftarrowtailfill@{#1}{#2}}
\makeatother

%THEOREMS
\newenvironment{Solution}
  {\ifsolutions\begin{tcolorbox}[colback=blue!8!white,colframe=blue!35!black,title=Solution]}
  {\end{tcolorbox}\fi}


\newcounter{booksection}
\theoremstyle{definition}
\newtheorem{problem}{Problem}
\newtheorem{bookproblem}{Problem}[booksection]
\newtheorem*{lemmma}{Lemma}
\newtheorem*{solution}{Solution}
\newtheorem*{fact}{Fact}
\newtheorem*{definition}{Definition}
\newtheorem*{definitions}{Definitions}
\theoremstyle{theorem}
\newtheorem{corollary}{Corollary}
\newtheorem{theorem}{Theorem}
\newcommand{\irr}{\text{irr}}
\newcommand{\ol}[1]{\overline{#1}}
%NEW COMMANDS
\newcommand{\mc}[1]{\mathcal{#1}}
\newcommand{\bb}[1]{\mathbb{#1}}
\newcommand{\bbm}[1]{\mathbbm{#1}}
\newcommand{\ms}[1]{\mathscr{#1}}
\newcommand{\ttt}[1]{\texttt{#1}}
\newcommand{\includecode}[2][Python]{\lstinputlisting[caption=#2, escapechar=, style=custom#1]{#2}}
\newcommand{\embf}[1]{\textbf{\emph{#1}}}
\newcommand{\tbf}[1]{\textbf{#1}}
	% TOPOLOGY COMMANDS
\newcommand{\intr}[1]{\accentset{\circ}{#1}}
\newcommand{\bndr}{\partial}
\newcommand{\clsr}[1]{\overline{#1}}
\newcommand{\tpl}[1][]{\mathscr{T}_{#1}}
	% COMPLEX VARIABLE COMMANDS
\newcommand{\Log}{\text{Log}}
\newcommand{\partials}[2]{\frac{\partial #1}{\partial #2}}
\newcommand{\hessian}[3]{\left(\begin{array}{cc}\frac{\partial^2 #1}{\partial #2^2} & \frac{\partial^2 #1}{\partial #1 \partial #2} \\ \frac{\partial^2 u}{\partial #1\partial #2} & \frac{\partial^2 #1}{\partial #2^2}\end{array}\right)}
\newcommand{\Res}{\text{Res}}
    % LINEAR ALGEBRA COMMANDS
\newcommand{\Span}{\text{span}}
\newcommand{\Null}{\text{Null}}
\newcommand{\Rank}{\text{rank}}
\newcommand{\Mat}[2]{\,\text{Mat}_{{#1}\times{#2}}}
\renewcommand{\u}{\vec u}
\renewcommand{\v}{\vec v}
\newcommand{\w}{\vec w}
\newcommand{\x}{\vec x}
\newcommand{\y}{\vec y}
\newcommand{\z}{\vec z}
\newcommand{\im}{\text{im}}
    % ALGEBRA COMMANDS
\newcommand{\Stab}[2]{\,\text{Stab}_{#1}({#2})}
\newcommand{\Cent}[2]{\,\text{C}_{#1}({#2})}
\newcommand{\Center}[1]{\,\text{Z}({#1})}
\newcommand{\Norm}[2]{\,\text{N}_{#1}({#2})}
\newcommand{\subgp}{\leq}
\newcommand{\Orb}[1]{\,{#1}\text{-orbit}}
\newcommand{\orbit}[2]{\,\text{orbit}_{#1}({#2})}
\newcommand{\Inn}[1]{\,\text{Inn}({#1})}
\newcommand{\Aut}[1]{\,\text{Aut}({#1})}
\newcommand{\Syl}{\,\text{Syl}}
\newcommand{\normal}{\trianglelefteq}
    % OTHER
\newcommand{\owl}{\widehat{\dbinom{\odot_\text{v}\odot}{\wr}}}

\allsectionsfont{\centering \normalfont\scshape} % Make all sections centered, the default font and small caps

\pagestyle{fancyplain} % Makes all pages in the document conform to the custom headers and footers
\fancyhead{} % No page header - if you want one, create it in the same way as the footers below
\fancyfoot[L]{} % Empty left footer
\fancyfoot[C]{} % Empty center footer
\fancyfoot[R]{\thepage} % Page numbering for right footer
\renewcommand{\headrulewidth}{0pt} % Remove header underlines
\renewcommand{\footrulewidth}{0pt} % Remove footer underlines
\setlength{\headheight}{13.6pt} % Customize the height of the header

\numberwithin{equation}{section} % Number equations within sections (i.e. 1.1, 1.2, 2.1, 2.2 instead of 1, 2, 3, 4)
\numberwithin{figure}{section} % Number figures within sections (i.e. 1.1, 1.2, 2.1, 2.2 instead of 1, 2, 3, 4)
\numberwithin{table}{section} % Number tables within sections (i.e. 1.1, 1.2, 2.1, 2.2 instead of 1, 2, 3, 4)

\setlength\parindent{0pt} % Removes all indentation from paragraphs - comment this line for an assignment with lots of text
\newcommand{\horrule}[1]{\rule{\linewidth}{#1}} % Create horizontal rule command with 1 argument of height

