%%%%%%%%%%%%%%%%%%%%%%%%%%%%%%%%%%%%%%%%%
% Short Sectioned Assignment
% LaTeX Template
% Version 1.0 (5/5/12)
%
% This template has been downloaded from:
% http://www.LaTeXTemplates.com
%
% Original author:
% Frits Wenneker (http://www.howtotex.com)
%
% License:
% CC BY-NC-SA 3.0 (http://creativecommons.org/licenses/by-nc-sa/3.0/)
%
%%%%%%%%%%%%%%%%%%%%%%%%%%%%%%%%%%%%%%%%%


%	PACKAGES AND OTHER DOCUMENT CONFIGURATIONS
%----------------------------------------------------------------------------------------

\documentclass[fontsize=11pt]{scrartcl} % A4 paper and 11pt font size

\usepackage[utf8]{inputenc}
\usepackage[T1]{fontenc} % Use 8-bit encoding that has 256 glyphs
\usepackage[english]{babel} % English language/hyphenation
\usepackage{amsmath,amsfonts,amsthm} % Math packages
\usepackage{sectsty} % Allows customizing section commands
\usepackage{mathtools}
\usepackage{graphicx}
\usepackage[all,cmtip]{xy}
\usepackage{centernot}
\usepackage{color}
\usepackage{bbm}
\usepackage{amssymb}
\usepackage{enumerate}
\usepackage{tcolorbox}
\usepackage{fancyhdr} % Custom headers and footers
\usepackage{geometry}
\usepackage{mathpazo}

\newif\ifsolutions
\solutionstrue

%\geometry{left=20mm,right=20mm,top=20mm}
%\geometry{a5paper, left=8mm,right=8mm,top=15mm,bottom=25mm}

%%% A5 paper size
% \geometry{papersize={168mm, 210mm}}% , left=20mm,right=20mm,top=30mm,bottom=40mm}
%%% Customized Paper size
%\geometry{papersize={180mm, 210mm}, left=28mm,right=28mm,top=15mm,bottom=25mm}

\newcommand{\cat}[1]{\mathbf{{#1}}}
\newcommand{\note}[1]{{\small \textit{\textbf{Note:} {#1}}}}
\def\C{\cat{C}}
\def\D{\cat{D}}
\def\E{\cat{E}}
\def\F{\cat{F}}
\def\Sets{\cat{Sets}}
\def\Grp{\cat{Grp}}
\def\Mon{\cat{Mon}}
\def\Ab{\cat{Ab}}
\def\Pos{\cat{Pos}}
\def\Cat{\cat{Cat}}
\def\Top{\cat{Top}}
\def\0{\cat{0}}
\def\1{\cat{1}}
\def\2{\cat{2}}
\def\3{\cat{3}}
\def\singleton{\{*\}}
\def\op{\textnormal{op}}



\def\H{\mathbb H}
\def\Q{\mathbb Q}
\def\R{\mathbb R}
\def\Z{\mathbb Z}
\def\Re{\textbf{Re}}
\def\Im{\textbf{Im}}
\def\id{\mathbbm{1}}

%%% ARROWS
\def\epi{\twoheadrightarrow}
\def\ipe{\twoheadleftarrow}
\def\mono{\rightarrowtail}
\def\onom{\rightarrowtail}
\def\xto{\xrightarrow}
\def\xinclusion{\xhookrightarrow}
\makeatletter
\providecommand*{\twoheadrightarrowfill@}{%
  \arrowfill@\relbar\relbar\twoheadrightarrow
}
\providecommand*{\twoheadleftarrowfill@}{%
  \arrowfill@\twoheadleftarrow\relbar\relbar
}
\providecommand*{\rightarrowtailfill@}{%
  \arrowfill@\relbar\relbar\rightarrowtail
}
\providecommand*{\leftarrowtailfill@}{%
  \arrowfill@\leftarrowtail\relbar\relbar
}
\newcommand*{\doublerightarrow}[2]{\mathrel{
  \settowidth{\@tempdima}{$\scriptstyle#1$}
  \settowidth{\@tempdimb}{$\scriptstyle#2$}
  \ifdim\@tempdimb>\@tempdima \@tempdima=\@tempdimb\fi
  \mathop{\vcenter{
    \offinterlineskip\ialign{\hbox to\dimexpr\@tempdima+1em{##}\cr
    \rightarrowfill\cr\noalign{\kern.5ex}
    \rightarrowfill\cr}}}\limits^{\!#1}_{\!#2}}}
\newcommand{\xepi}[2][]{\ext@arrow 0359\twoheadrightarrowfill@{#1}{#2}}
\newcommand{\xipe}[2][]{\ext@arrow 0359\twoheadleftarrowfill@{#1}{#2}}
\newcommand{\xmono}[2][]{\ext@arrow 0359\rightarrowtailfill@{#1}{#2}}
\newcommand{\xonom}[2][]{\ext@arrow 0359\leftarrowtailfill@{#1}{#2}}
\makeatother

%THEOREMS
\newenvironment{Solution}
  {\ifsolutions\begin{tcolorbox}[colback=blue!8!white,colframe=blue!35!black,title=Solution]}
  {\end{tcolorbox}\fi}


\newcounter{booksection}
\theoremstyle{definition}
\newtheorem{problem}{Problem}
\newtheorem{bookproblem}{Problem}[booksection]
\newtheorem*{lemmma}{Lemma}
\newtheorem*{solution}{Solution}
\newtheorem*{fact}{Fact}
\newtheorem*{definition}{Definition}
\newtheorem*{definitions}{Definitions}
\theoremstyle{theorem}
\newtheorem{corollary}{Corollary}
\newtheorem{theorem}{Theorem}
\newcommand{\irr}{\text{irr}}
\newcommand{\ol}[1]{\overline{#1}}
%NEW COMMANDS
\newcommand{\mc}[1]{\mathcal{#1}}
\newcommand{\bb}[1]{\mathbb{#1}}
\newcommand{\bbm}[1]{\mathbbm{#1}}
\newcommand{\ms}[1]{\mathscr{#1}}
\newcommand{\ttt}[1]{\texttt{#1}}
\newcommand{\includecode}[2][Python]{\lstinputlisting[caption=#2, escapechar=, style=custom#1]{#2}}
\newcommand{\embf}[1]{\textbf{\emph{#1}}}
\newcommand{\tbf}[1]{\textbf{#1}}
	% TOPOLOGY COMMANDS
\newcommand{\intr}[1]{\accentset{\circ}{#1}}
\newcommand{\bndr}{\partial}
\newcommand{\clsr}[1]{\overline{#1}}
\newcommand{\tpl}[1][]{\mathscr{T}_{#1}}
	% COMPLEX VARIABLE COMMANDS
\newcommand{\Log}{\text{Log}}
\newcommand{\partials}[2]{\frac{\partial #1}{\partial #2}}
\newcommand{\hessian}[3]{\left(\begin{array}{cc}\frac{\partial^2 #1}{\partial #2^2} & \frac{\partial^2 #1}{\partial #1 \partial #2} \\ \frac{\partial^2 u}{\partial #1\partial #2} & \frac{\partial^2 #1}{\partial #2^2}\end{array}\right)}
\newcommand{\Res}{\text{Res}}
    % LINEAR ALGEBRA COMMANDS
\newcommand{\Span}{\text{span}}
\newcommand{\Null}{\text{Null}}
\newcommand{\Rank}{\text{rank}}
\newcommand{\Mat}[2]{\,\text{Mat}_{{#1}\times{#2}}}
\renewcommand{\u}{\vec u}
\renewcommand{\v}{\vec v}
\newcommand{\w}{\vec w}
\newcommand{\x}{\vec x}
\newcommand{\y}{\vec y}
\newcommand{\z}{\vec z}
\newcommand{\im}{\text{im}}
    % ALGEBRA COMMANDS
\newcommand{\Stab}[2]{\,\text{Stab}_{#1}({#2})}
\newcommand{\Cent}[2]{\,\text{C}_{#1}({#2})}
\newcommand{\Center}[1]{\,\text{Z}({#1})}
\newcommand{\Norm}[2]{\,\text{N}_{#1}({#2})}
\newcommand{\subgp}{\leq}
\newcommand{\Orb}[1]{\,{#1}\text{-orbit}}
\newcommand{\orbit}[2]{\,\text{orbit}_{#1}({#2})}
\newcommand{\Inn}[1]{\,\text{Inn}({#1})}
\newcommand{\Aut}[1]{\,\text{Aut}({#1})}
\newcommand{\Syl}{\,\text{Syl}}
\newcommand{\normal}{\trianglelefteq}
    % OTHER
\newcommand{\owl}{\widehat{\dbinom{\odot_\text{v}\odot}{\wr}}}

\allsectionsfont{\centering \normalfont\scshape} % Make all sections centered, the default font and small caps

\pagestyle{fancyplain} % Makes all pages in the document conform to the custom headers and footers
\fancyhead{} % No page header - if you want one, create it in the same way as the footers below
\fancyfoot[L]{} % Empty left footer
\fancyfoot[C]{} % Empty center footer
\fancyfoot[R]{\thepage} % Page numbering for right footer
\renewcommand{\headrulewidth}{0pt} % Remove header underlines
\renewcommand{\footrulewidth}{0pt} % Remove footer underlines
\setlength{\headheight}{13.6pt} % Customize the height of the header

\numberwithin{equation}{section} % Number equations within sections (i.e. 1.1, 1.2, 2.1, 2.2 instead of 1, 2, 3, 4)
\numberwithin{figure}{section} % Number figures within sections (i.e. 1.1, 1.2, 2.1, 2.2 instead of 1, 2, 3, 4)
\numberwithin{table}{section} % Number tables within sections (i.e. 1.1, 1.2, 2.1, 2.2 instead of 1, 2, 3, 4)

\setlength\parindent{0pt} % Removes all indentation from paragraphs - comment this line for an assignment with lots of text
\newcommand{\horrule}[1]{\rule{\linewidth}{#1}} % Create horizontal rule command with 1 argument of height



\title{%
\normalfont\normalsize 
\textsc{Umass Amherst} \\ [25pt] % Your university, school and/or department name(s)
\horrule{0.5pt} \\[0.4cm] % Thin top horizontal rule
\huge Fun with Cats \\[0.2cm] % The assignment title
\large\textit{An Introduction to Archery}\\
\horrule{2pt} \\%[-1.5cm]
}
\author{Ben Kushigian} % Your name
\date{} % Today's date or a custom date
\begin{document}
\maketitle % Print the title

\section*{What is Category Theory?}
Category theory, lovingly\footnote{and accurately} referred to as abstract
nonsense, can initially be thought of as the ``study of abstract algebras of
functions''. Category theory isn't interested in the values being mapped by
functions so much as the general structure induced by the functions
themselves.\\ 

For example, the set theorist would write injectivity of a function \(f: A \to
B\) as

\[\forall xy \in A: f(x) = f(y) \implies x = y\]

while the category theorist would say that \(f\) is injective\footnote{This
construction is called a \textit{monomorphism} in category theory land, and by
reversing the arrows 

\[\xymatrix{  D & B\ar@<.5ex>[l]^j\ar@<-.5ex>[l]_i  & A\ar[l]_f }.\]

we get an \textit{epimorphism} that is equivalent to surjectivity of functions.
This `reversing the arrows' trick is called the dual relation and is prevalent
in mathematics: union is dual to intersection, addition is dual to
multiplication, logical or is dual to logical and, etc\ldots}
if for all \(g,h : C \to A\), \(fg = fh\) entails \(g = h\),

\[\xymatrix{  C \ar@<.5ex>[r]^g\ar@<-.5ex>[r]_h & A \ar[r]^f & B}.\]

This is non-constructive, but allows us to easily generalize our definition.
Why, you ask?  Well, we haven't explicitly used any elements and, in fact, we
can restate this without mentioning that we are using sets and functions at all:
all we have used are some names \(A\), \(B\), \(C\), some arrows
\(\xrightarrow{f}\), \(\xrightarrow{g}\), and \(\xrightarrow{h}\), and
equality, as well as related them in some way (\(C\xrightarrow{g} A\), \ldots). \\

While this may seem an arbitrary distinction at first, it turns out that this
abstraction allows us to apply definitions to many different classes of objects
since we never had to reference elements or types directly.  What we have done
is captured a relationship between `objects' and `arrows' that can be applied to
many different mathematical types, from sets and functions to groups and
group homomorphisms to topological spaces and continuous functions to
logical statements and logical entailment to types and subtyping to lambdas and
evaluation. This is the power of category theory.  The computer scientist should
immediately be reminded of parametric polymorphism.\\

To make this generalization work, however, we need to relax our definition
slightly. Instead of talking about \textit{functions between sets} we instead
talk about \textit{arrows\footnote{arrows are also referred to as
\textit{morphisms}, as in group/ring/graph homomorphisms} between objects}. At
first blush arrows and objects can be thought of as strange names for functions
and sets --- this provides good intuition and ends up being correct for sets.\\

A collection of arrows and objects satisfying some rather modest properties is
called a \textit{category}, and these can be thought of as a \textit{type} of
mathematical object. As we can see, we are really just talking about PL!\\

\section*{This Reading Group}
This summer we will be working through Steve Awodey's ``Category Theory'' which
provides a lovely introduction to the subject with relatively few mathematical
prerequisites. Mathematical constructions will be presented as
needed\footnote{In particular, Awodey develops to structures that are very
familiar to computer scientists: the monoid and the poset. Monoids are things
that can be added and that have an identity. Strings under concatenation are
your basic ``free monoid'', though the natural numbers under addition, \(\Z_n\),
and matrix multiplication also offer examples of monoids.\\

Posets, or partially ordered sets, are another common construction in computer
science. A DAG (directed acyclic graph), for example, induces a poset under the
relation \(u \leq b\) if \(u = v\) or \(REACH(u,v)\) (this is just the
transitive closure) and is used, for example, in dataflow analysis.} and a large
number of examples are presented from a variety of subjects, including
mathematics, mathematical logic, and programming languages. From the preface:

\begin{quote}
    {\it Why write a new textbook on Category Theory, when we already have Mac
    Lane's \textnormal{Categories for the Working Mathematician?}. Simply put,
    because Mac Lane's book is for the working (and aspiring) mathematician.
    What is needed now, after 30 years of spreading into various other
    disciplines and places in the curriculum, is a book for everyone else.

    [\ldots]

    The students in my courses often have little background in Mathematics
    beyond a course in Discrete Math and some Calculus or Linear Algebra or a
    course or two in Logic. Nonetheless, eventually, as researchers in Computer
    Science or Logic, many will need to be familiar with the basic notions of
    Category Theory, without the benefit of much further mathematical
    training. [\ldots] Most of my students do not know what a free group is
    (yet), and so they are not illuminated to learn that it is an example of an
    adjoint.

    This, then, is intended as a text and reference book on Category Theory, not
    only for students of Mathematics, but also for researchers and students in
    Computer Science, Logic, Linguistics, Cognitive Science, Philosophy, and any
    of the other fields that now make use of it.}
\end{quote}

In a perfect world we would cover natural transformations, adjoints, and  the
Yoneda lemma, and wrap up by covering monads, as these mark the end of
``elementary category theory.'' However, this is perhaps too ambitious and I
will be satisfied if we come away with enough material and background to finish
the text on our own time. Regardless, there is a lot of material so we will try
to set a brisk pace, but since this will be somewhat a case of the blind leading
the blind (I have only worked through the first few chapters myself, as well as
skimming through later sections) we will stop and take time as needed.\\

In my experience it is very easy to passively read a math text without learning
anything and the remedy for this is to do problems. To this end I will be
picking out problems for us to work through on our own, with the idea of either
presenting these once a week or swapping them to be `graded'. These are of
course optional (since this is not for credit) but I will urge people to at
least attempt some problems every week.\\

\subsection*{Meetings and Administratia}
I'm not sure when/where we will meet or how long or often. I would be open to
starting off with some lectures to give us a bit of a head start and then either
continue with the lecture format or switch to a round table discussion format.
We can talk about this when we have a first meeting.\\

\subsection*{Joining}
Anyone who is interested in joining or being affiliated in any way can reach out
on Slack (which I check annually) or email me at
\texttt{bkushigian@cs.umass.edu}.

\end{document}
